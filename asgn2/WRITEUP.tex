\documentclass[11pt]{article} % Document type
\usepackage{graphicx}
\usepackage{float}
\title{Assignment 2}
\author{Michael Kamensky}
\date{\today} % Sets the date to \today, or any date you specify
\begin{document}\maketitle % Start the document
\section{Running mathlib-test}
Here is an example of runnig my tests:

\begin{verbatim} 
mkamensk@vera:~/cse13s/asgn2$ ./mathlib-test -a -s
e() = 2.718281828459046, M_E = 2.718281828459045, diff = 0.000000000000000
e term = 17
pi_bbp() = 3.141592653589793, M_PI = 3.141592653589793, diff = 0.000000000000000
pi_bbp() term = 11
pi_viete() = 3.141592653589793, M_PI = 3.141592653589793, diff = -0.000000000000000
pi_viete() factors = 25
pi_euler() = 3.141592558095903, M_PI = 3.141592653589793, diff = -0.000000095493891
pi_euler() terms = 10000000
pi_madhava() = 3.141592653589800, M_PI = 3.141592653589793, diff = 0.000000000000007
pi_madhava() terms = 27
sqrt_newton() = 1.414213562373095, M_SQRT2 = 1.414213562373095, diff = -0.000000000000000
sqrt_newton() iteration = 6
\end{verbatim}
\section{Calculation of PI}
These three methods converge really quickly to desired value
\begin{figure}[H]
\begin{center}
\includegraphics[height=3in,width=5in]{pi.pdf}
\caption{This is Viete BBp and Madhava Pi Calculation Error}
\end{center}
\end{figure}

But Euler Pi calculation converges really slowely
\begin{figure}[H]
\begin{center}
\includegraphics[height=3in,width=5in]{pi_euler.pdf}
\caption{This is Euler Pi Calculation Error}
\end{center}
\end{figure}

\section{Calculation of E}
Shows the error clauclation of e()
\begin{figure}[H]
\begin{center}
\includegraphics[height=3in,width=5in]{e.pdf}
\caption{This is Euler E Calculation Error}
\end{center}
\end{figure}

\section{Calculation of Square Root of Two}
Shows the error clauclation of Sqrt 2
\begin{figure}[H]
\begin{center}
\includegraphics[height=3in,width=5in]{sqrt.pdf}
\caption{This is Square Root of Two Newton Calculation Error}
\end{center}
\end{figure}


\end{document}
